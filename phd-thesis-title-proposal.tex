% This is samplepaper.tex, a sample chapter demonstrating the
% LLNCS macro package for Springer Computer Science proceedings;
% Version 2.21 of 2022/01/12
%
\documentclass[runningheads]{llncs}
%
\usepackage[T1]{fontenc}
% T1 fonts will be used to generate the final print and online PDFs,
% so please use T1 fonts in your manuscript whenever possible.
% Other font encondings may result in incorrect characters.
%
\usepackage{graphicx}
\usepackage{acronym}
\usepackage[backend=biber,style=lncs]{biblatex}
\usepackage{cleveref}
% Used for displaying a sample figure. If possible, figure files should
% be included in EPS format.
%
% If you use the hyperref package, please uncomment the following two lines
% to display URLs in blue roman font according to Springer's eBook style:
%\usepackage{color}
%\renewcommand\UrlFont{\color{blue}\rmfamily}
%
\addbibresource{bibliography.bib}
\addbibresource{mypubblications.bib}

\acrodef{cas}[CAS]{Collective Adaptive Systems}
\acrodef{iot}[IoT]{Internet of Things}
\acrodef{ecc}[ECC]{Edge-Cloud Continuum}
\acrodef{rl}[RL]{Reinforcement Learning}

\begin{document}

%
\title{{\normalfont PhD Thesis Proposal}\\\vspace{0.1em}Engineering Collective Systems in the Wearable Edge-Cloud Continuum: Models and Platforms
%\thanks{Supported by organization x.}
}
%
\titlerunning{PhD Thesis Proposal -- Cycle 39}
% If the paper title is too long for the running head, you can set
% an abbreviated paper title here
%
\author{Nicolas Farabegoli\inst{1}\orcidID{0000-0002-7321-358X}\\Cycle 39 -- Computer Science and Engineering} %\inst{1}\orcidID{0000-0002-7321-358X}}
%
%\authorrunning{F. Author et al.}
% First names are abbreviated in the running head.
% If there are more than two authors, 'et al.' is used.
%
\institute{
    Alma Mater Studiorum -- Università di Bologna, Cesena (FC) 47522, Italy
    \email{nicolas.farabegoli@unibo.it}
}
%
\maketitle              % typeset the header of the contribution
%
\begin{abstract}
% The abstract should briefly summarize the contents of the paper in
% 150--250 words.
The Edge-Cloud Continuum (ECC) has emerged as a solution to the limitations of traditional cloud systems by enabling a more flexible and scalable infrastructure that spans multiple computational tiers,
from IoT devices to the cloud.
%
Several approaches have been proposed to leverage the ECC in IoT applications and Collective Adaptive Systems (CAS);
however,
they are often tailored to specific use cases,
focusing on isolated aspects such as latency,
energy efficiency,
and system performance.
%
This makes them difficult to generalize and reuse across different contexts.
%
This PhD thesis aims to develop a general and comprehensive framework to support the design and deployment of the next generation of IoT and CAS in the ECC,
providing a general model specifically designed at the intersection of other research areas:
self-organisation principles for defining the collective and collaborative behavior of the system,
multi-tier programming models for handling the heterogeneity of the infrastructure producing sound system specifications,
and AI-driven techniques like reinforcement learning for dynamic resource management,
predictive scaling,
and dynamic system reconfiguration.

% \keywords{First keyword  \and Second keyword \and Another keyword.}
\end{abstract}

\section{Introduction}
\label{sec:introduction}
The rapid proliferation of interconnected devices in environments such as smart cities,
IoT ecosystems,
wearable devices,
and swarm robotics has introduced new challenges in system design and management.
%
These systems,
characterized by numerous devices with varying capabilities and constraints,
must interact cooperatively and adaptively in dynamic environments.
%
Traditional cloud-based solutions,
limited by latency and bandwidth constraints,
are increasingly being replaced by the \ac{ecc}~\cite{DBLP:journals/access/MoreschiniPLNHT22},
which enables distributed computing across multiple tiers.

This research aims to develop innovative approaches for engineering \ac{cas}~\cite{DBLP:conf/birthday/BucchiaroneM19} within the \ac{ecc}.
%
The proposed framework will support the entire \ac{cas} workflow,
from design and deployment to runtime management and adaptation.
%
By leveraging self-organization principles,
the framework will ensure coherent collective behavior,
while multitier~\cite{DBLP:journals/csur/WeisenburgerWS20} programming models will address the heterogeneity and layered characteristics of modern infrastructures.
%
Additionally,
AI techniques,
including reinforcement learning,
will enhance dynamic resource management,
predictive scaling,
and system reconfiguration,
nowadays essential aspects in the \ac{ecc}.

Building on foundational work in \emph{aggregate computing}~\cite{DBLP:journals/computer/BealPV15} and pulverization~\cite{DBLP:journals/fi/CasadeiPPVW20,DBLP:journals/iotj/CasadeiFPPSV22},
this research extends existing paradigms to enable scalable,
deployment-independent,
highly reconfigurable,
adaptable systems by opening these approaches to a wider spectrum of collective heterogeneous applications in the \ac{ecc}.
%
The aggregate computing approach offers a high-level abstraction for modeling complex collective system behavior through computational fields~\cite{DBLP:journals/tocl/AudritoVDPB19},
while pulverization partitions execution models into independent components that preserve functional behavior by achieving flexibility and deployment independence.
%
By integrating AI-driven reconfiguration strategies,
the framework aims to optimize system performance and resource utilization in real time,
coping with the dynamic nature of the \ac{ecc}.

This research is supported by the Italian PRIN project ``CommonWears'' (2020HCWWLP)\footnote{\url{https://common-wears.github.io/2022/}},
which aims to develop novel models and architectures for next-generation community-oriented wearable computing systems.

The manuscript is structured as follows:
\Cref{sec:research-gap} identifies the research gap,
\Cref{sec:background} provides background information,
\Cref{sec:research-activities} outlines research activities and outcomes achieved so far,
\Cref{sec:publications} discusses academic contributions.
%
\Cref{sec:future-directions} concludes with future research directions and long-term goals.

% This manuscript presents the research activities carried out during the first year of my PhD,
% and outlines the future research directions of my research.

% The main goal of my PhD is to develop and investigate innovative engineering approaches to develop and deploy \ac{cas}~\cite{DBLP:conf/birthday/BucchiaroneM19} in emergent and dynamic environments,
% like the \ac{ecc}~\cite{DBLP:journals/access/MoreschiniPLNHT22}.
% %
% With this research line,
% I aim to provide a general but effective framework supporting the entire workflow of \ac{cas},
% from the design models,
% to the deployment and runtime management of the system,
% up to the system's reconfiguration and adaptation to changing conditions.

% Smart cities,
% \ac{iot} applications,
% swarm robotics,
% exhibit the common trait of being composed of many interconnected devices,
% each with its own capabilities and constraints,
% and that must \emph{cooperatively} interact,
% while adapting coherently as a whole to dynamic environments.
% %
% The new opportunities offered by the \ac{ecc} infrastructures paved the way for the development of new applications and services,
% but they shift the focus from the traditional two tier vision to a more complex,
% multitier~\cite{DBLP:journals/csur/WeisenburgerWS20} heterogeneous environment.

% The \emph{aggregate computing} approach~\cite{DBLP:journals/computer/BealPV15} to \ac{cas} engineering,
% comprises a \emph{macro-programming} model and language~\cite{DBLP:journals/csur/Casadei23} based on computational fields~\cite{DBLP:journals/tocl/AudritoVDPB19},
% where the system's behavior is described in terms of functional composition of these fields.
% %
% This paradigm provides a high-level of abstraction to engineer \ac{cas},
% and has been successfully applied to a wide range of applications,
% from smart cities to swarm robotics,
% assuming homogeneous fully peer-to-peer networks.
% %
% Pulverization~\cite{DBLP:journals/fi/CasadeiPPVW20} defines a partitioning model for neatly separate the self-organising behavior of the system from the actual deployment.
% %
% This approach preserves the same functional behavior of the original system,
% enabling the system to be deployed in different infrastructures opening the door to the \ac{ecc}.
% %
% This foundational work represents the starting point of my research activities,
% founded by the Italian PRIN project ``CommonWears'' (2020HCWWLP).

% With this thesis,
% I aim to extend the current state of the art by providing a comprehensive framework to support the design and deployment of \ac{cas} in the \ac{ecc} with focus on communities of wearables,
% leveraging the principles of self-organisation to define the collective and collaborative behavior of the system,
% multi-tier programming models to handle the heterogeneity of the infrastructure producing sound system specifications,
% and AI-driven techniques like reinforcement learning for dynamic resource management,
% predictive scaling,
% and dynamic system reconfiguration.

% The manuscript is structured as follows:
% TBD.

\subsection{Research Gap}
\label{sec:research-gap}

In the context of the \ac{ecc},
various approaches have been proposed to overcome the limitations of traditional cloud systems.
%
However,
most of these solutions adopt a bottom-up perspective,
focusing on specific use cases or isolated aspects such as latency,
energy efficiency,
or system performance.
%
Consequently,
they often lack the flexibility and scalability required for broader applicability across diverse environments,
such as smart cities,
wearable ecosystems,
and swarm robotics.
%
For example,
a latency-optimized solution for smart city infrastructure may struggle to adapt when applied to a dynamic swarm of autonomous drones.

Another key challenge lies in modeling the collective and collaborative behavior of interconnected devices within the \ac{ecc}.
%
Current methodologies often fail to capture emergent behaviors that arise when devices interact as a system.
%
This limitation calls for engineering models capable of describing system behavior in terms of both individual device actions and collective outcomes,
ensuring coherence and adaptability across different operational contexts.

Additionally,
the dynamic nature of the ECC requires mechanisms for real-time resource management and system reconfiguration to maintain performance and efficiency under changing conditions.
%
Existing solutions,
often based on static rules or manual configurations, struggle to scale as system complexity increases.
%
\ac{rl} and other AI-driven techniques offer promising solutions by enabling systems to learn optimal reconfiguration strategies through experience.
%
However,
current applications of RL are typically tailored to specific scenarios,
limiting their generalizability and scalability across diverse system architectures.

This research aims to address these gaps by developing a comprehensive framework that integrates collective behavior modeling,
multitier programming models,
and AI-driven reconfiguration techniques.
%
By ensuring deployment independence,
scalability,
and adaptability,
the framework will provide a unified approach to engineering next-generation \ac{cas} within the \ac{ecc}.

\section{Background}
\label{sec:background}

\subsection{Edge-Cloud Continuum}
\label{sec:ecc}

The increasing proliferation of \ac{iot} devices and latency-sensitive applications has led to the evolution of computing paradigms beyond traditional centralized cloud models.
%
The concept of the \ac{ecc} emerges as a response to the need for balancing computation between cloud data centers and edge devices,
aiming to enhance performance,
reduce latency,
and improve resource utilization.

In traditional cloud computing,
large-scale data centers offer vast computational resources and storage capabilities.
%
However,
the physical distance between these data centers and end devices introduces latency,
which is unacceptable for applications such as autonomous vehicles,
smart healthcare,
and industrial automation~\cite{DBLP:journals/iotj/ShiCZLX16}.
%
Edge computing addresses this limitation by bringing computation closer to the data sources,
enabling faster response times and reducing bandwidth usage~\cite{DBLP:journals/computer/Satyanarayanan17}.
%
Nevertheless,
edge devices often have limited computational and storage capacities.

The edge-cloud continuum integrates both paradigms,
creating a seamless orchestration of resources across different layers of the infrastructure.
%
This continuum allows for dynamic workload distribution,
where latency-critical tasks are executed at the edge,
while compute-intensive processes are offloaded to the cloud~\cite{DBLP:series/sci/BonomiMNZ14}.
%
Additionally,
intermediate layers such as fog computing provide a middle ground,
further reducing latency and enabling localized data processing~\cite{DBLP:journals/ccr/GonzalezR14}.

Recent advancements focus on adaptive resource allocation,
where applications can migrate dynamically between edge and cloud environments based on network conditions,
resource availability,
and application requirements~\cite{DBLP:journals/fgcs/RomanLM18}.
%
Such adaptability is essential for collective systems that must respond to changing conditions in real-time.

\subsection{Collective Adaptive Systems}
\label{sec:cas}

\subsection{Aggregate Computing}
\label{sec:aggregate-computing}

\subsection{Pulverization}
\label{sec:pulverization}

\subsection{Multitier}
\label{sec:multitier}

\section{Research Activities and Outcomes}
\label{sec:research-activities}

The first year of my PhD has been focused on the extension and development of approaches to foster \emph{deployment independence} of Collective Adaptive Systems (CAS)~\cite{DBLP:conf/birthday/BucchiaroneM19} in emergent and dynamic environments like the Edge-cloud Continuum~\cite{DBLP:journals/access/MoreschiniPLNHT22}.
%
During this research activities,
I was able to publish and submit two relevant papers in high-quality journals,
and published and presented many papers in conferences and workshops.
%
Moreover,
I've been involved in tutoring activities and attended several PhD courses and school to improve my research skills and knowledge.

\subsection{PhD Scope}

\subsection{Scientific Contributions and Activities}

My first year started with a natural progression of my Master's thesis,
where I developed a practical framework to implement the pulverisation model~\cite{DBLP:journals/fi/CasadeiPPVW20,DBLP:journals/iotj/CasadeiFPPSV22}--
a modern approach to partition the execution model of CAS into independent components--
and extended it to support dynamic components' relocation over the infrastructure at runtime.
%
The proposed framework has been validated through a set of experiments,
showing the feasibility of the approach and the benefits in terms of deployment independence and non-functional properties,
such as cost,
power consumption,
and performance.

Subsequently,
I investigate new reconfiguration approaches shifting from local reconfiguration rules to a global perspective,
where the system's reconfiguration is expressed by a collective program,
leveraging \emph{Aggregate Computing}.
%
In this contribution,
I proposed a general pulverisation framework architecture where dynamic components' reconfiguration is captured as a first-class citizen,
and I proposed a novel (collective) algorithm to balance the system's load during the reconfiguration process.
%
The proposed approach has been validated through a set of experiments,
showing that the proposed algorithm can better adapt to disruptive and changing conditions,
and it can provide better performance and stability compared to allocations defined a priori.

Then,
a semi-formalisation of a novel approach to support the pulverisation component's relocation based on Reinforcement Learning (RL) and Graph Neural Networks (GNN) has been developed.
%
This vision aims to pave the way for the development of intelligent CAS in modern infrastructures,
where the system's reconfiguration is driven by AI-based techniques.

An orthogonal contribution w.r.t. the pulverisation model has been developed to partition the program specification of CAS into independent functions (or services).
%
This approach aims to provide a more flexible and fine-grained approach to partition the system's program specification,
and it has been validated through a set of experiments,
showing both empirically and formally that the proposed approach preserves the same functional behavior of the original (monolithic deployed) system,
and it can provide non-functional benefits in terms of power consumption and performance.

Finally,
I've been involved in other works and collaborations with other PhD students and researchers
in the development and evaluation of coordination algorithms for UAVs,
and in the development of a novel federated learning approach.
%
All these activities can be framed in my research direction opening new research perspectives and collaborations,
and they have been beneficial to my research growth and knowledge.


% During the first year of my PhD,
% I focused on developing innovative models to enhance the deployment independence of collective adaptive systems (CAS).
% %
% Building upon previous work on pulverisation~\cite{DBLP:journals/iotj/CasadeiFPPSV22,DBLP:journals/fi/CasadeiPPVW20}--a modern approach to partitioning the execution model of CAS into independent components--
% I proposed a practical framework that implements and extends the pulverisation model to support dynamic components' relocation over the infrastructure at runtime.

% To validate the proposed framework and the extended model with reconfiguration rules,
% I conducted a series of experiments.
% %
% These experiments demonstrated the feasibility of the approach and highlighted its benefits in terms of deployment independence and non-functional properties,
% such as cost,
% power consumption,
% and performance.
% %
% The results of this research have been published in a paper submitted to the \emph{Future Generation Computer Systems} journal~\cite{DBLP:journals/fgcs/FarabegoliPCV24}.

% During the first year of my PhD,
% I have been working on the development of innovative models fostering the deployment independence of collective adaptive systems (CAS).

% In particular,
% based on previous work on the pulverisation~\cite{DBLP:journals/iotj/CasadeiFPPSV22,DBLP:journals/fi/CasadeiPPVW20} --
% a modern approach to partition into independent components the execution model of CAS --
% I proposed a practical framework implementing the pulverisation model,
% and extending it to support the dynamic reconfiguration at runtime.
% %
% The proposed framework and the extended model with reconfiguration rules have been validated through a set of experiments,
% showing the feasibility of the approach and the benefits in terms of deployment independence and non-functional properties,
% like cost, power consumption, and performance.
% %
% The results of the experiments have been published in a paper submitted to the journal \emph{Future Generation Computer Systems journal}~\cite{DBLP:journals/fgcs/FarabegoliPCV24}.

% Subsequently,
% I have been working on the development of a novel approach to better support the dynamic reconfiguration of CAS based on the pulverisation model:
% pre-defined reconfiguration rules may not be sufficient to cover all possible scenarios,
% and they may lead to suboptimal solutions,
% like oscillator behaviors.
% %
% To address this issue,
% I proposed to shift the perspective from local reconfiguration rules to a global perspective,
% where the system's reconfiguration is expressed by a collective program,
% leveraging \emph{Aggregate Computing}.
% %
% In this contribution,
% I proposed a general pulverisation framework architecture where dynamic components' reconfiguration is captured as a first-class citizen,
% and I proposed a novel (collective) algorithm to balance the system's load during the reconfiguration process.
% %
% The proposed approach has been validated through a set of experiments,
% showing that the proposed algorithm can better adapt to disruptive and changing conditions,
% and it can provide better performance and stability compared to allocations defined a priori.
% %
% The results of the experiments and the proposed approach have been submitted and currently under review to the journal \emph{Internet of Things; Engineering Cyber Physical Human Systems}~\cite{farabegoli4798700dynamic}.

% Finally,
% I worked on the development of a novel approach to partition the program specification of CAS into independent functions (or services) to achieve flexible deployment in the \emph{Edge-cloud Continuum}.
% %
% This contribution is orthogonal to the pulverisation model,
% and it aims to provide a more flexible and fine-grained approach to partition the system's program specification.
% %
% The proposed model has been formalised via an operational semantics,
% and properties like self-stabilisation preservation, and deployment independence have been formally proved.
% %
% The new model has been validated through a set of experiments,
% showing empirically that the proposed approach preserves the same functional behavior of the original (monolithic deployed) system,
% and it can provide non-functional benefits in terms of power consumption and performance by paying a small overhead in terms of communication.
% %
% The results of the experiments and the proposed approach have been accepted and presented at the \emph{ACSOS 2024} conference~\cite{DBLP:conf/acsos/FarabegoliFAVE24}.

% Aside from my main research activities,
% I've been involved in other works and collaborations with other PhD students and researchers,
% in the development and evaluation of coordination algorithms for UAVs,
% and in the development of a novel federated learning approach.
% %
% All these activities can be framed in my research direction opening new research perspectives and collaborations,
% and they have been beneficial to my research growth and knowledge.

\subsection{Publications}
\label{sec:publications}

As a result of my research activities,
I've published and submitted the following papers:

\subsubsection{Accepted and Published}
\begin{refsection}[mypubblications.bib]
    % \DeclareFieldFormat{labelnumberwidth}{}
    \nocite{*} % Include all references from file1.bib only
    \printbibliography[heading=none]
\end{refsection}

\subsubsection{Submitted and Under Review}
\begin{enumerate}
    \item Submission to the \emph{ACM Transactions on Autonomous and Adaptive Systems}
\end{enumerate}

% \begin{enumerate}

%     \item Nicolas Farabegoli, Danilo Pianini, Roberto Casadei, \& Mirko Viroli (2024). Scalability through Pulverisation: Declarative deployment reconfiguration at runtime. Future Gener. Comput. Syst., 161, 545–558.    
%     \item Farabegoli, N., Pianini, D., Casadei, R., \& Viroli, M. (2024). Dynamic Iot Deployment Reconfiguration: A Global-Level Self-Organisation Approach. Internet of Things Journal, Elsevier (submitted).
%     \item Davide Domini, Nicolas Farabegoli, Gianluca Aguzzi, \& Mirko Viroli (2024). Towards Intelligent Pulverized Systems: a Modern Approach for Edge-Cloud Services. In Proceedings of the 25th Workshop "From Objects to Agents", Bard (Aosta), Italy, July 8-10, 2024 (pp. 233–251). CEUR-WS.org.
%     \item Nicolas Farabegoli, Mirko Viroli, \& Roberto Casadei (2024). Flexible Self-organisation for the Cloud-Edge Continuum: a Macro-programming Approach. In IEEE International Conference on Autonomic Computing and Self-Organizing Systems, ACSOS 2024, Denmark, Aarhus, September 16-20, 2024. IEEE (to appear).
%     \item Denys Grushchak, Jenna Kline, Danilo Pianini, Nicolas Farabegoli, Martina Baiardi, Gianluca Aguzzi, \& Christopher Stewart (2024). Decentralized Multi-Drone Coordination for Wildlife Video Acquisition. In IEEE International Conference on Autonomic Computing and Self-Organizing Systems, ACSOS 2024, Denmark, Aarhus, September 16-20, 2024. IEEE (to appear).   
%     \item Davide Domini, Gianluca Aguzzi, Nicolas Farabegoli, Mirko Viroli, \& Lukas Esterle (2024). Opportunistic Fully-Distributed Federated Learning. In IEEE International Conference on Autonomic Computing and Self-Organizing Systems, ACSOS 2024, Denmark, Aarhus, September 16-20, 2024. IEEE (to appear).
%     \item Denys Grushchak, Jenna Kline, Danilo Pianini, \& Nicolas Farabegoli (2024). An Agent-based Model of Directional Multi-herds. In IEEE International Conference on Autonomic Computing and Self-Organizing Systems, ACSOS 2024, Denmark, Aarhus, September 16-20, 2024. IEEE (to appear).
%     \item Nicolas Farabegoli (2024). Intelligent Pulverised Collective-adaptive Systems. In IEEE International Conference on Autonomic Computing and Self-Organizing Systems, ACSOS 2024, Denmark, Aarhus, September 16-20, 2024. IEEE (to appear).
% \end{enumerate}

\subsection{Conferences and Workshops}

During this first year of my PhD,
I've attended the following conferences and workshops:

\begin{itemize}
    \item Attended the \emph{International ABS Workshop 2023};
    \item Presented a paper at the \emph{Workshop ``From Objects to Agents'' 2024};
    \item Attended \emph{ACM SIGPLAN International Conference on Functional Programming};
    \item Presented 2 out of 5 papers at the \emph{IEEE International Conference on Autonomic Computing and Self-Organizing Systems - ACSOS 2024}.
\end{itemize}

% \subsection{Research Goals}

% \subsection{Literature Review}

\section{Academic Activities}

\subsection{Tutoring Activities}

In this first year of my PhD,
I have been involved in tutoring activities for the course \textbf{PROGRAMMAZIONE AD OGGETTI [cod. 70219]} for the Bachelor's Degree in Computer Science and Engineering for the academic year 2023/2024,
and currently involved in the academic year 2024/2025 edition of the same course.

\subsection{Doctoral School and Courses}

I've attended the \textbf{Bertinoro International Spring School 2024} in March 2024,
in particular I've attended the \emph{Graph Neural Network} course by Prof. Fabrizio Silvestri,
the \emph{Program analysis: from proving correctness to proving incorrectness} course by Prof. Roberto Bruni and Robera Gori,
and the \emph{Large Language Models} course by Prof. Danilo Croce.
%
Moreover,
I've attended the following PhD courses:
\begin{enumerate}
    \item \emph{How to Write and Publish a Research Paper in Computer Science and Engineering} by Prof. Zeynep Kiziltan;
    \item \emph{Risk Assessment of Machine Learning for Cybersecurity} by Prof. Fabio Pierazzi;
    \item \emph{Introduction to complex systems science} by Prof. Andrea Roli;
\end{enumerate}
At the end of the first year,
I've obtained \textbf{15} CFU with evaluation, and \textbf{4} CFU without evaluation.

% \section{Self Evaluation}

% I strongly believe that the first year of my PhD has been very productive and successful.
% %
% Despite initial difficulties in getting my works accepted in conferences and journals,
% I've been able to publish and submit three papers in high-quality venues,
% and I've been able to collaborate with other researchers and PhD students in different research areas.
% %
% During this year,
% I've been able to improve my research skills,
% and I've been able to learn new techniques and methodologies to address research problems,
% get familiar with new research tools and environments,
% and improve my writing and presentation skills.

\section{Future Directions}
\label{sec:future-directions}

I plan to continue my research on developing innovative models for engineering collective adaptive systems.
%
My focus will be on integrating techniques that support comprehensive
and effective reconfiguration strategies while exploring multi-tier programming models to facilitate the deployment of CAS in the Edge-Cloud Continuum.
%
In this context,
I aim to investigate the use of capability-based mechanisms for controlling system deployment and runtime reconfiguration,
ensuring that system specifications are correct by construction leveraging advanced type systems.
%
Additionally,
I am interested in integrating AI-based techniques to support runtime reconfiguration while preserving the functional correctness of the system,
ensured at the design time by the type system.


%
% ---- Bibliography ----
%
% BibTeX users should specify bibliography style 'splncs04'.
% References will then be sorted and formatted in the correct style.
%
% \nocite{*}
% \bibliographystyle{splncs04}
% \bibliography{bibliography,mypubblications}
\printbibliography

\end{document}
