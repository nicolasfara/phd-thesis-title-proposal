% This is samplepaper.tex, a sample chapter demonstrating the
% LLNCS macro package for Springer Computer Science proceedings;
% Version 2.21 of 2022/01/12
%
\documentclass[runningheads]{llncs}
%
\usepackage[T1]{fontenc}
% T1 fonts will be used to generate the final print and online PDFs,
% so please use T1 fonts in your manuscript whenever possible.
% Other font encondings may result in incorrect characters.
%
\usepackage{graphicx}
% Used for displaying a sample figure. If possible, figure files should
% be included in EPS format.
%
% If you use the hyperref package, please uncomment the following two lines
% to display URLs in blue roman font according to Springer's eBook style:
%\usepackage{color}
%\renewcommand\UrlFont{\color{blue}\rmfamily}
%

\begin{document}

%
\title{PhD Cycle 39 --- First Year Final Report
%\thanks{Supported by organization x.}
}
%
%\titlerunning{Abbreviated paper title}
% If the paper title is too long for the running head, you can set
% an abbreviated paper title here
%
\author{Nicolas Farabegoli\inst{1}\orcidID{0000-0002-7321-358X}}
%
%\authorrunning{F. Author et al.}
% First names are abbreviated in the running head.
% If there are more than two authors, 'et al.' is used.
%
\institute{
    Alma Mater Studiorum -- Università di Bologna, Cesena FC 47522, Italy
    \email{nicolas.farabegoli@unibo.it}
}
%
\maketitle              % typeset the header of the contribution
%
% \begin{abstract}
% The abstract should briefly summarize the contents of the paper in
% 150--250 words.

% \keywords{First keyword  \and Second keyword \and Another keyword.}
% \end{abstract}
%
%
%
\section{Research Activities and Outcomes}

The first year of my PhD has been focused on the extension and development of approaches to foster \emph{deployment independence} of Collective Adaptive Systems (CAS)~\cite{DBLP:conf/birthday/BucchiaroneM19} in emergent and dynamic environments like the Edge-cloud Continuum~\cite{DBLP:journals/access/MoreschiniPLNHT22}.
%
During this research activities,
I was able to publish and submit two relevant papers in high-quality journals,
and published and presented many papers in conferences and workshops.
%
Moreover,
I've been involved in tutoring activities and attended several PhD courses and school to improve my research skills and knowledge.

\subsection{PhD Scope}

The main goal of my PhD is to develop and investigate innovative engineering approaches to develop and deploy CAS.
%
More specifically,
I aim to develop models and techniques to simplify the deployment of collective systems in emergent and dynamic environments,
taking as a reference the Edge-cloud Continuum~\cite{DBLP:journals/access/MoreschiniPLNHT22}.

The main goal is to provide a comprehensive and effective framework to fully exploit edge-cloud resources to support the execution of modern CAS,
and to provide runtime support for the dynamic reconfiguration and adaptation of the system to changing conditions,
dealing with device heterogeneity and network instability.

AI-based techniques may be employed to support the system's reconfiguration preserving self-organisation and self-stabilisation properties,
and to provide effective way to manage the system's reconfiguration at runtime by devising policies and strategies otherwise difficult to implement with traditional approaches.


The research activities are framed in the context of the \emph{Aggregate Computing} paradigm~\cite{DBLP:journals/computer/BealPV15},
a novel approach to engineer CAS based on the concept of \emph{computational fields},
where the system's behavior emerges from the interaction of autonomous and distributed components.

\subsection{Scientific Contributions and Activities}

My first year started with a natural progression of my Master's thesis,
where I developed a practical framework to implement the pulverisation model~\cite{DBLP:journals/fi/CasadeiPPVW20,DBLP:journals/iotj/CasadeiFPPSV22}--
a modern approach to partition the execution model of CAS into independent components--
and extended it to support dynamic components' relocation over the infrastructure at runtime.
%
The proposed framework has been validated through a set of experiments,
showing the feasibility of the approach and the benefits in terms of deployment independence and non-functional properties,
such as cost,
power consumption,
and performance.

Subsequently,
I investigate new reconfiguration approaches shifting from local reconfiguration rules to a global perspective,
where the system's reconfiguration is expressed by a collective program,
leveraging \emph{Aggregate Computing}.
%
In this contribution,
I proposed a general pulverisation framework architecture where dynamic components' reconfiguration is captured as a first-class citizen,
and I proposed a novel (collective) algorithm to balance the system's load during the reconfiguration process.
%
The proposed approach has been validated through a set of experiments,
showing that the proposed algorithm can better adapt to disruptive and changing conditions,
and it can provide better performance and stability compared to allocations defined a priori.

Then,
a semi-formalisation of a novel approach to support the pulverisation component's relocation based on Reinforcement Learning (RL) and Graph Neural Networks (GNN) has been developed.
%
This vision aims to pave the way for the development of intelligent CAS in modern infrastructures,
where the system's reconfiguration is driven by AI-based techniques.

An orthogonal contribution w.r.t. the pulverisation model has been developed to partition the program specification of CAS into independent functions (or services).
%
This approach aims to provide a more flexible and fine-grained approach to partition the system's program specification,
and it has been validated through a set of experiments,
showing both empirically and formally that the proposed approach preserves the same functional behavior of the original (monolithic deployed) system,
and it can provide non-functional benefits in terms of power consumption and performance.

Finally,
I've been involved in other works and collaborations with other PhD students and researchers
in the development and evaluation of coordination algorithms for UAVs,
and in the development of a novel federated learning approach.
%
All these activities can be framed in my research direction opening new research perspectives and collaborations,
and they have been beneficial to my research growth and knowledge.


% During the first year of my PhD,
% I focused on developing innovative models to enhance the deployment independence of collective adaptive systems (CAS).
% %
% Building upon previous work on pulverisation~\cite{DBLP:journals/iotj/CasadeiFPPSV22,DBLP:journals/fi/CasadeiPPVW20}--a modern approach to partitioning the execution model of CAS into independent components--
% I proposed a practical framework that implements and extends the pulverisation model to support dynamic components' relocation over the infrastructure at runtime.

% To validate the proposed framework and the extended model with reconfiguration rules,
% I conducted a series of experiments.
% %
% These experiments demonstrated the feasibility of the approach and highlighted its benefits in terms of deployment independence and non-functional properties,
% such as cost,
% power consumption,
% and performance.
% %
% The results of this research have been published in a paper submitted to the \emph{Future Generation Computer Systems} journal~\cite{DBLP:journals/fgcs/FarabegoliPCV24}.

% During the first year of my PhD,
% I have been working on the development of innovative models fostering the deployment independence of collective adaptive systems (CAS).

% In particular,
% based on previous work on the pulverisation~\cite{DBLP:journals/iotj/CasadeiFPPSV22,DBLP:journals/fi/CasadeiPPVW20} --
% a modern approach to partition into independent components the execution model of CAS --
% I proposed a practical framework implementing the pulverisation model,
% and extending it to support the dynamic reconfiguration at runtime.
% %
% The proposed framework and the extended model with reconfiguration rules have been validated through a set of experiments,
% showing the feasibility of the approach and the benefits in terms of deployment independence and non-functional properties,
% like cost, power consumption, and performance.
% %
% The results of the experiments have been published in a paper submitted to the journal \emph{Future Generation Computer Systems journal}~\cite{DBLP:journals/fgcs/FarabegoliPCV24}.

% Subsequently,
% I have been working on the development of a novel approach to better support the dynamic reconfiguration of CAS based on the pulverisation model:
% pre-defined reconfiguration rules may not be sufficient to cover all possible scenarios,
% and they may lead to suboptimal solutions,
% like oscillator behaviors.
% %
% To address this issue,
% I proposed to shift the perspective from local reconfiguration rules to a global perspective,
% where the system's reconfiguration is expressed by a collective program,
% leveraging \emph{Aggregate Computing}.
% %
% In this contribution,
% I proposed a general pulverisation framework architecture where dynamic components' reconfiguration is captured as a first-class citizen,
% and I proposed a novel (collective) algorithm to balance the system's load during the reconfiguration process.
% %
% The proposed approach has been validated through a set of experiments,
% showing that the proposed algorithm can better adapt to disruptive and changing conditions,
% and it can provide better performance and stability compared to allocations defined a priori.
% %
% The results of the experiments and the proposed approach have been submitted and currently under review to the journal \emph{Internet of Things; Engineering Cyber Physical Human Systems}~\cite{farabegoli4798700dynamic}.

% Finally,
% I worked on the development of a novel approach to partition the program specification of CAS into independent functions (or services) to achieve flexible deployment in the \emph{Edge-cloud Continuum}.
% %
% This contribution is orthogonal to the pulverisation model,
% and it aims to provide a more flexible and fine-grained approach to partition the system's program specification.
% %
% The proposed model has been formalised via an operational semantics,
% and properties like self-stabilisation preservation, and deployment independence have been formally proved.
% %
% The new model has been validated through a set of experiments,
% showing empirically that the proposed approach preserves the same functional behavior of the original (monolithic deployed) system,
% and it can provide non-functional benefits in terms of power consumption and performance by paying a small overhead in terms of communication.
% %
% The results of the experiments and the proposed approach have been accepted and presented at the \emph{ACSOS 2024} conference~\cite{DBLP:conf/acsos/FarabegoliFAVE24}.

% Aside from my main research activities,
% I've been involved in other works and collaborations with other PhD students and researchers,
% in the development and evaluation of coordination algorithms for UAVs,
% and in the development of a novel federated learning approach.
% %
% All these activities can be framed in my research direction opening new research perspectives and collaborations,
% and they have been beneficial to my research growth and knowledge.

\subsection{Publications}

As a result of my research activities,
I've published and submitted the following papers:

\begin{enumerate}
    \item Nicolas Farabegoli, Danilo Pianini, Roberto Casadei, \& Mirko Viroli (2024). Scalability through Pulverisation: Declarative deployment reconfiguration at runtime. Future Gener. Comput. Syst., 161, 545–558.    
    \item Farabegoli, N., Pianini, D., Casadei, R., \& Viroli, M. (2024). Dynamic Iot Deployment Reconfiguration: A Global-Level Self-Organisation Approach. Internet of Things Journal, Elsevier (submitted).
    \item Davide Domini, Nicolas Farabegoli, Gianluca Aguzzi, \& Mirko Viroli (2024). Towards Intelligent Pulverized Systems: a Modern Approach for Edge-Cloud Services. In Proceedings of the 25th Workshop "From Objects to Agents", Bard (Aosta), Italy, July 8-10, 2024 (pp. 233–251). CEUR-WS.org.
    \item Nicolas Farabegoli, Mirko Viroli, \& Roberto Casadei (2024). Flexible Self-organisation for the Cloud-Edge Continuum: a Macro-programming Approach. In IEEE International Conference on Autonomic Computing and Self-Organizing Systems, ACSOS 2024, Denmark, Aarhus, September 16-20, 2024. IEEE (to appear).
    \item Denys Grushchak, Jenna Kline, Danilo Pianini, Nicolas Farabegoli, Martina Baiardi, Gianluca Aguzzi, \& Christopher Stewart (2024). Decentralized Multi-Drone Coordination for Wildlife Video Acquisition. In IEEE International Conference on Autonomic Computing and Self-Organizing Systems, ACSOS 2024, Denmark, Aarhus, September 16-20, 2024. IEEE (to appear).   
    \item Davide Domini, Gianluca Aguzzi, Nicolas Farabegoli, Mirko Viroli, \& Lukas Esterle (2024). Opportunistic Fully-Distributed Federated Learning. In IEEE International Conference on Autonomic Computing and Self-Organizing Systems, ACSOS 2024, Denmark, Aarhus, September 16-20, 2024. IEEE (to appear).
    \item Denys Grushchak, Jenna Kline, Danilo Pianini, \& Nicolas Farabegoli (2024). An Agent-based Model of Directional Multi-herds. In IEEE International Conference on Autonomic Computing and Self-Organizing Systems, ACSOS 2024, Denmark, Aarhus, September 16-20, 2024. IEEE (to appear).
    \item Nicolas Farabegoli (2024). Intelligent Pulverised Collective-adaptive Systems. In IEEE International Conference on Autonomic Computing and Self-Organizing Systems, ACSOS 2024, Denmark, Aarhus, September 16-20, 2024. IEEE (to appear).
\end{enumerate}

\subsection{Conferences and Workshops}

During this first year of my PhD,
I've attended the following conferences and workshops:

\begin{enumerate}
    \item \emph{International ABS Workshop 2023} -- attended;
    \item \emph{Workshop ``From Objects to Agents'' 2024} -- 1 paper accepted and presented;
    \item \emph{ACM SIGPLAN International Conference on Functional Programming} -- attended;
    \item \emph{IEEE International Conference on Autonomic Computing and Self-Organizing Systems - ACSOS 2024} -- 5 papers accepted and 2 papers presented;
\end{enumerate}

% \subsection{Research Goals}

% \subsection{Literature Review}

\section{Academic Activities}

\subsection{Tutoring Activities}

In this first year of my PhD,
I have been involved in tutoring activities for the course \textbf{PROGRAMMAZIONE AD OGGETTI [cod. 70219]} for the Bachelor's Degree in Computer Science and Engineering for the academic year 2023/2024,
and currently involved in the academic year 2024/2025 edition of the same course.

\subsection{Doctoral School and Courses}

I've attended the \textbf{Bertinoro International Spring School 2024} in March 2024,
in particular I've attended the \emph{Graph Neural Network} course by Prof. Fabrizio Silvestri,
the \emph{Program analysis: from proving correctness to proving incorrectness} course by Prof. Roberto Bruni and Robera Gori,
and the \emph{Large Language Models} course by Prof. Danilo Croce.
%
Moreover,
I've attended the following PhD courses:
\begin{enumerate}
    \item \emph{How to Write and Publish a Research Paper in Computer Science and Engineering} by Prof. Zeynep Kiziltan;
    \item \emph{Risk Assessment of Machine Learning for Cybersecurity} by Prof. Fabio Pierazzi;
    \item \emph{Introduction to complex systems science} by Prof. Andrea Roli;
\end{enumerate}
At the end of the first year,
I've obtained \textbf{15} CFU with evaluation, and \textbf{4} CFU without evaluation.

\section{Self Evaluation}

I strongly believe that the first year of my PhD has been very productive and successful.
%
Despite initial difficulties in getting my works accepted in conferences and journals,
I've been able to publish and submit three papers in high-quality venues,
and I've been able to collaborate with other researchers and PhD students in different research areas.
%
During this year,
I've been able to improve my research skills,
and I've been able to learn new techniques and methodologies to address research problems,
get familiar with new research tools and environments,
and improve my writing and presentation skills.

\section{Next Year Plan}

For the next year,
I plan to continue my research activities on the development of innovative models for engineering collective adaptive systems,
focusing on the integration of techniques to support comprehensive and effective reconfiguration strategies,
and investigate the integration of multi-tier programming models to support the deployment of CAS in the Edge-cloud Continuum.

%
% ---- Bibliography ----
%
% BibTeX users should specify bibliography style 'splncs04'.
% References will then be sorted and formatted in the correct style.
%
% \nocite{*}
\bibliographystyle{splncs04}
\bibliography{bibliography}

\end{document}
